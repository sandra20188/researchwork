\documentclass[11pt]{book}              % Book class in 11 points
\parindent0pt  \parskip10pt             % make block paragraphs
\raggedright                            % do not right justify

\title{\bf HOW COLOR AFFECTS PEOPLES MOODS}    % Supply information
\author{ Nabawesi sandra specious 11/u/11594/eve 211015135}
\date{\today}                           %   Use current date. 

% Note that book class by default is formatted to be printed back-to-back.
\begin{document}                        % End of preamble, start of text.
\frontmatter                            % only in book class (roman page #s)
\maketitle                              % Print title page.
                  
\tableofcontents

\section{INTRODUCTION}
Colors may just seem simple and unimportant, but they affect our daily lives more than we may know. If someone is feeling angry, it could just be because they’re angry, or it could be perhaps that they are surrounded by or looking at the color red. That’s right! People’s moods can change just because they are looking at different colors! There are many theories on how just a simple color can change one’s whole mood.

 According to Johnson, color does affect mood by producing certain chemicals and stimulating different feelings such as hunger. For example, blue can make one feel calm because it releases calming chemicals, and red can make one hungry because it is an appetite stimulant. Yellow can make one feel irritated, and it is a fact that people lose their temper most in yellow rooms. However, pink is tranquilizing and can make one feel weak. In conclusion, Johnson says that depending on the color, one’s body can do things (like producing chemicals) that cause a certain emotional reaction (mad, sad, etc.). 

Another idea, by Smith, is that the effect color produces is based on what one’s body does in response. For example, yellow is mentally stimulating, and activates memory, whereas red increases confidence. Also, brown can make a person feel orderly and stable, while a dark blue can make one feel sad. Therefore, Smith says that different colors do in fact change one’s mood and the consequences can be negative or positive. 

A third writer, Wollard, seems to think that color can affect one’s mood, but the effect also can depend on one’s culture and what one’s personal reflection may be. For example, someone from Japan may not associate red with anger, as people from the U.S. tend to do. Also, a person who likes the color brown may associate brown with happiness. However, Wollard does think that colors can make everyone feel the same, or close to the same, mood. According to Wollard, pink reduces aggression, which is why the walls of the jail cells in the Seattle prison are pink! Also, brown can make one feel comforted. Wollard feels that colors do affect one’s mood, but there are other factors that can alter what one is supposed to feel. 

Eric, John, and Paraag’s , main point about color psychology is that color has both a physiological and psychological effect. For example, green makes people feel relaxed because it relaxes their muscles and makes them breathe deeper and more slowly. Furthermore, blue lowers blood pressure, which makes one feel calm. Eric, John, and Paraag conclude that color affects one’s mood because of what it does to the body.



\section{Analysis}
There are three ideas about color psychology in these sources, and they all say that color affects one’s mood. They differ based on what factors influence the effects of color, such as culture, opinion, and what goes on inside one’s body. One of the three ideas is that color affects mood based on one’s personal opinions. For example, if a person dislikes the color pink, he may associate pink with hate. Another idea states that color affects mood based on one’s culture. For example, someone from the Uganda may think of the color green when referring to envy, while people in Kenya think of yellow in connection with wanting what someone else has. However, the majority of the sources consulted say that color affects mood by influencing what goes on inside of people. For example, seeing the color blue releases calming chemicals, which in turn makes one calm. Also, because yellow is the hardest color for the eye to focus on, people may become irritated when looking at yellow, and it is a proven fact that babies cry most in yellow nurseries. These theories do not seem to have much in common.
\section{Conclusion}
Color does affect one’s mood, but it can affect boys and girls differently. For example research says girls find green neutral and balancing but boys find it secure and safe. However, there are also some similarities between the girls 
and boys. For example, both think brown is boring and also, both think that pink made them feel love and affection. Furthermore, from research it is known that yellow would make boys and girls feel irritated, but most boys and girls reported that it made them feel happy and cheery. Overall, most of the results were different from the research sources consulted. In any case, it is obvious that colors have a great effect on one’s mood.

\section{ Reference}
 refer airey ,D. (2006). How does color psychology work? Retrieved
October 19, 2007, Eric, John, and Paraag. (2007). Color psychology. Retrieved
October 19, 2007,


                                       


\end{document}                          % The required last line
